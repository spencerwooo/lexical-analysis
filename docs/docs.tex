\documentclass[UTF8]{ctexart}
\usepackage[a4paper, top=25.4mm, bottom=25.4mm, left=31.8mm, right=31.8mm]{geometry}
\usepackage{graphicx}
\usepackage{amsmath}
\usepackage{multirow}
\usepackage{verbatim}
\usepackage{subcaption}
\usepackage{tabu}
\usepackage{booktabs}
\usepackage{minted}
\usemintedstyle{manni}
\usepackage[table]{xcolor}

\setlength{\parskip}{1em}
\definecolor{lightergray}{gray}{0.95}
\setlength{\tabcolsep}{12pt}
\renewcommand{\arraystretch}{1}

\begin{document}
\begin{titlepage}
  \begin{center}
    \vspace*{1cm}

    \Large
    编译原理

    \vspace{0.5cm}
    \Huge
    \textbf{词法分析实验实验报告}

    \vfill

    \normalsize\kaishu
    班级:07111603 \\
    学号:1120161730 \\
    姓名:武上博 \\
    \today
    \vspace{1cm}
  \end{center}
\end{titlepage}

\tableofcontents
\newpage

\section{实验目的}
\begin{enumerate}
  \item 熟悉 C 语言的词法规则,了解编译器词法分析器的主要功能
  \item 掌握典型词法分析器构造的相关技术和方法,设计并实现 C 语言词法分析器
  \item 掌握编译器从前端到后端各个模块的工作原理,词法分析模块与其他模块之间的交互过程
\end{enumerate}

\section{实验内容}
根据 C 语言的词法规则,设计并识别 C 语言所有单词类的词法分析器的确定有限状态自动机,并使用 Java、C/C++、Python 其中的任意一种语言,采用程序中心法或者数据中心法设计并实现词法分析器。词法分析器的输入为 C 语言源程序,输出为属性字流。

\section{实验的具体过程步骤}
\subsection{程序实现的大致思路}
为了和接下来语法分析模块相配合,本次实现的词法分析器接受 C 语言源程序作为输入,利用 XML 作为格式进行输出分析的词法内容。同时,为了和 BIT-MiniCC 进行更好的整合,本次实验我决定使用 Python 作为主语言进行各个模块的实现。

经过分析,我觉得本次实验中词法分析器是如下图 \ref{fig:figure1} 的大致构造:

\begin{figure}[h]
  \includegraphics[width=\linewidth]{images/lexical.png}
  \caption{词法分析器的大致流程}
  \label{fig:figure1}
\end{figure}

也就是说,我们本次需要实现的模块有:

\begin{enumerate}
  \item 文件读入
  \item 程序预处理模块(去除每行前部分空格、注释等)
  \item 进行词法分析,依次识别:
  \begin{itemize}
    \item 标识符 Identifier
    \item 常量 Constant
    \begin{itemize}
        \item 整数型常量 Integer Constant
        \item 点型常量 Floating Constant
    \end{itemize}
    \item 字符 Char
    \item 字符串 String
    \item 算符 Punctuation
      \begin{itemize}
        \item 运算符 Operator
        \item 界限符 Delimiter
      \end{itemize}
  \end{itemize}
  \item 输出 XML 文件
\end{enumerate}

\subsection{具体模块的实现}

接下来,我们分别对各个模块相应的具体实现方法进行介绍。

\subsubsection{程序输入和预处理}

本次实验中的输入是一个 C 语言程序的源文件。我们从命令行读入需要处理的文件路径,处理文件内容。

\begin{minted}[linenos,frame=lines,framesep=2mm]{python}
def main():
  # Print usage if arguments are not legal
  if len(sys.argv) < 2:
    print('[Usage] ./scan.py <C source file path>')
    sys.exit(0)

  # Read file from file path taken from command line arguments
  filePath = sys.argv[1]
  with open(filePath, 'r', encoding='utf-8') as f:
    content = f.readlines()
\end{minted}

在读文件时,我使用了 \texttt{readlines()} 函数为了逐行读入文件。我们得到 \texttt{content},也就是代码的基本内容。之后,我们对读入的内容进行预处理。

\begin{minted}[linenos,frame=lines,framesep=2mm]{python}
# 主函数内的内容,预处理代码内容
code = preProcess(content)
# 预处理函数
def preProcess(content):
  code = ''
  # Trim leading white space 去掉每行最前面的空白
  for line in content:
    if line != '\n':
      code = code + line.lstrip()
    else:
      code = code + line
  return code
\end{minted}

我首先定义代码变量 \texttt{code},之后按行处理代码内容,对于每一行代码,如果代码不是空行,那么我就将这一行的代码和前面定义的 \texttt{code} 相连接,之后我们只需要处理 \texttt{code} 缓冲区内的代码内容即可。

接下来,我们利用自动机对输入字符串进行匹配来判断其输入类型。首先,我定义了下面一个指针(即当前读入字符位置)和五个识别类型:

\begin{minted}[linenos,frame=lines,framesep=2mm]{python}
# 指针查找位置
index = 0

# Token 属性
codeNum = 1
codeType = ''
codeLine = 1
codeValue = ''
codeValid = 0
\end{minted}

我维护指针 \texttt{index} 用来遍历输入代码串,利用 \texttt{codeNum}、\texttt{codeType}、\texttt{codeLine}、\\\texttt{codeValue}、\texttt{codeValid} 来分别标识:当前识别的 Token 数量、当前识别 Token 的种类、当前读到代码行数、当前识别 Token 的内容以及当前识别 Token 是否合法。

之后,我构造 \texttt{scanner()} 来对输入串进行扫描识别处理:

\begin{minted}[linenos,frame=lines,framesep=2mm]{python}
def scanner(code):
  # 当前扫描代码位置
  global index
  # 当前识别符数
  global codeNum
  # 当前代码行
  global codeLine

  # 识别到词语的类别
  global codeType
  codeType = ''
  # 识别到的词语
  global codeValue
  codeValue = ''
  # 当前识别字符
  character = code[index]
  index = index + 1

  # Ignore white space
  while character == ' ':
    character = code[index]
    index = index + 1
  ...
\end{minted}

在主函数 \texttt{main()} 中,我通过这样的方式调用扫描器:

\begin{minted}[linenos,frame=lines,framesep=2mm]{python}
# Start scanning!
global codeNum
while index <= len(code) - 1:
  scanner(code)
\end{minted}

接下来,我构建了五个自动机,分别对标识符、常量、字符、字符串和算符进行了识别。

\subsubsection{标识符 Identifier 的判断}
标识符 Identifier 是由字母、数字或下划线“\_”组成的,具体的定义大致是这样的:

\begin{equation}
\begin{split}
  identifier \rightarrow &\ identifier-nondigit \ \\ |\ & identifier\ identifier-nondigit \ \\ |\ & identifier\ digit
\end{split}
\end{equation}
\begin{equation}
  identifier-nondigit \rightarrow \ nondigit \ |\ universal-character
\end{equation}
\begin{equation}
  nondigit \rightarrow \_ \ |\ a ... z\ |\ A ... Z
\end{equation}
\begin{equation}
  digit \rightarrow 0 ... 9
\end{equation}

为了识别标识符,我确定如图 \ref{fig:figure2} 的状态机。

\begin{figure}[h]
  \centering
  \includegraphics[width=0.45\textwidth]{images/identifier.png}
  \caption{识别标识符的状态机}
  \label{fig:figure2}
\end{figure}

之后,我们就可以实现对标识符的识别:

\begin{minted}[linenos,frame=lines,framesep=2mm]{python}
# Identifier!
if character.isalpha() or character == '_':
  while character.isalpha() or character.isdigit() or character == '_':
    codeValue = codeValue + character
    character = code[index]
    index = index + 1
  codeType = 'identifier'
  index = index - 1
\end{minted}

在识别了标识符之后,我们可以直接继续判断这个标识符是不是 C 语言中的关键词(Keyword)之一。我们本次实验需要判断的是 C 语言的子集,需要进行识别的关键词有这些:

\begin{center}
  \captionsetup{position=above}
  \captionof{table}{C 语言关键词表}
  \begin{tabular}{c|c|c|c|c}
    \hline
    auto & break & case & char & const \\
    continue & default & do & double & else \\
    enum & extern & float & for & goto \\
    if & inline & int & long & register \\
    restrict & return & short & signed & sizeof \\
    static & struct & switch & typedef & union \\
    unsigned & void & volatile & while \\
    \hline
  \end{tabular}
\end{center}

于是,我们维护一个关键词列表 \texttt{cKeywords},之后通过字符串匹配的方式识别标识符是否为关键词:

\begin{minted}[linenos,frame=lines,framesep=2mm]{python}
# Keyword!
for keyword in cKeywords:
  if codeValue == keyword:
    codeType = 'keyword'
    break
\end{minted}

这样我们就可以成功的识别 Token 中的标识符和关键词。

\subsubsection{常量(整形常量 Integer Constant 和浮点型常量 Floating Constant)的判断}
在 C 语言的语法中,常量 Constant 有整形和浮点型两种需要进行识别。整形常量 Integer Constant 的文法可以这样描述:

\begin{equation}
  \begin{split}
    integer-constant \rightarrow \ decimal-constant \ integer-suffix\\|\ octal-constant \ integer-suffix\\|\ hexadecimal-constant \ integer-suffix
  \end{split}
\end{equation}
\begin{equation}
  decimal-constant \rightarrow 1...9\ |\ decimal-constant\ digit
\end{equation}
\begin{equation}
  octal-constant \rightarrow 0\ |\ octal-constant\ octal-digit
\end{equation}
\begin{equation}
  \begin{split}
    hexadecimal-constant \rightarrow hexadecimal-prefix\ hexadecimal-digit\\|\ hexadecimal-constant\ hexadecimal-digit
  \end{split}
\end{equation}
\begin{equation}
  hexadecimal-prefix \rightarrow 0x\ |\ 0X
\end{equation}
\begin{equation}
  \begin{split}
    integer-suffix \rightarrow\ unsigned\ long\ |\ unsigned\ long-long\\|\ long\ unsigned\ |\ long-long\ unsigned
  \end{split}
\end{equation}

浮点型常量 Floating Constant 的文法可以这样描述:

\begin{equation}
  floating-constant \rightarrow\ decimal-floating-constant\ |\ hexadecimal-floating-constant
\end{equation}
\begin{equation}
\begin{split}
    decimal-floating-constant \rightarrow\ fractional-constant\ exp\ floating-suffix\ \\ |\ digital-seq\ exp\ floating-suffix
\end{split}
\end{equation}
\begin{equation}
\begin{split}
    hexadecimal-floating-constant \rightarrow\ hexadecimal-prefix\ \\hexadecimal-frac-constant\ binary-exp\ floating-suffix\ \\|\ hexadecimal-prefix\ hexadecimal-digit-seq\\\ binary-exp\ floating-suffix
\end{split}
\end{equation}
\begin{equation}
  exp \rightarrow\ e\ digit-seq\ |\ E\ digit-seq
\end{equation}
\begin{equation}
  sign \rightarrow\ +\ |\ -
\end{equation}
\begin{equation}
  digit-seq \rightarrow\ digit\ |\ digit-seq\ digit
\end{equation}
\begin{equation}
\begin{split}
  hexadecimal-frac-constant \rightarrow\ hexadecimal-digit-seq\ .\ \\hexadecimal-digit-seq\ |\ hexadecimal-digit-seq\ .
\end{split}
\end{equation}
\begin{equation}
  binary-exp \rightarrow\ p\ digit-seq\ |\ P\ digit-seq
\end{equation}
\begin{equation}
\begin{split}
    hexadecimal-digit-seq \rightarrow\ hexadecimal-digit-seq\ |\\\ hexadecimal-digit-seq\ hexadecimal-digit
\end{split}
\end{equation}
\begin{equation}
  floating-suffix \rightarrow\ f\ |\ F\ |\ l\ |\ L
\end{equation}

其中我们需要特别注意的是:
\begin{itemize}
  \item 整数常量中有十进制 Decimal、八进制 Octal 和十六进制 Hexadecimal 三种常量需要考虑,其中以 \texttt{0} 开头的是八进制数字,以 \texttt{0x} 或 \texttt{0X} 开头的是十六进制数字,其他我们都直接认为是十进制数字
  \item 浮点型常量中需要考虑科学计数法,比如 \texttt{1.5e-4} 就是一个合法的浮点型常量
  \item 整数常量中的后缀字符有 u、U 表示无符号整形 unsigned 和 l、L 表示长整型 long 或 long long,也就是说比如 \texttt{512ull} 这样的无符号长整型就是合法的十进制整型常量
  \item 浮点型常量中后缀字符有 f、F、l、L,其中 f、F 表示 float 类型的浮点型常量,没有 f、F 后缀的浮点型常量我们认为是 double 类型的;l、L 表示长浮点型常量,即比如 \texttt{1.5F}、\texttt{4.9L} 这样的浮点数是合法的
\end{itemize}

综合整形常量和浮点型常量的文法表示和注意事项,我们可以构造如下图 \ref{fig:figure3} 的自动机来识别整形和浮点型常量:

\begin{figure}[h]
  \centering
  \includegraphics[width=\linewidth]{images/constant.png}
  \caption{整形和浮点型常量识别自动机}
  \label{fig:figure3}
\end{figure}

之后,我们就可以按照状态机的进行代码实现了。

\begin{minted}[linenos,frame=lines,framesep=2mm]{python}
# Integer and float constants
elif character.isdigit():
  global constantState
  while character.isdigit() or character in '-.xXeEaAbBcCdDfFuUlL':
    codeValue = codeValue + character
    # 进入自动机,进行串匹配
    if constantState == 0:
      if character == '0':
        constantState = 1
    ...
    elif constantState == 13 or constantState == 15:
      if character:
        constantState = -1
    character = code[index]
    index = index + 1

  index = index - 1
  # 接受整形常量
  if constantState in (1, 2, 4, 9, 10, 11, 12, 13, 14):
    codeType = 'integer constant'
    constantState = 0
  # 接受浮点型常量
  elif constantState == 6 or constantState == 15:
    codeType = 'floating constant'
    constantState = 0
  # 被拒绝或未识别的常量
  else:
    codeType = 'illegal constant'
    constantState = 0
\end{minted}

这样,我们就基本实现了对整形和浮点型常量的识别。

\subsubsection{字符 Character、字符串 String 的判断}
字符和字符串我们需要分别进行判断。我在这一步骤的判断是基于界限符单引号\ ' 和双引号\ " 来判断 Token 属于字符还是字符串。

我们首先利用文法描述字符常量:
\begin{equation}
  char-constant \rightarrow\ 'c-char-seq'
\end{equation}
\begin{equation}
  c-char-seq \rightarrow\ c-char\ |\ c-char-seq\ c-char
\end{equation}
\begin{equation}
  c-char \rightarrow\ all\ symbols\ other\ than\ ', \ \setminus\ and\ \setminus n\ |\ esc-seq
\end{equation}
\begin{equation}
  esc-seq \rightarrow\ \setminus'\ |\ \setminus"\ |\ \setminus?\ |\ \setminus\setminus\ |\ \setminus a\ |\ \setminus b\ |\ \setminus f\ |\ \setminus n\ |\ \setminus r\ |\ \setminus t\ |\ \setminus v
\end{equation}

字符串常量同样可以用文法表示如下:
\begin{equation}
  string-literal \rightarrow\ "s-char-seq"
\end{equation}
\begin{equation}
  s-char-seq \rightarrow\ s-char\ |\ s-char-seq\ s-char
\end{equation}
\begin{equation}
  s-char \rightarrow\ all\ symbols\ other\ than\ ', \ \setminus\ and\ \setminus n\ |\ esc-seq
\end{equation}
\begin{equation}
  esc-seq \rightarrow\ \setminus'\ |\ \setminus"\ |\ \setminus?\ |\ \setminus\setminus\ |\ \setminus a\ |\ \setminus b\ |\ \setminus f\ |\ \setminus n\ |\ \setminus r\ |\ \setminus t\ |\ \setminus v
\end{equation}

根据上面的文法描述,我们可以画出识别字符和字符串的自动机如下图 \ref{fig:figure4} 所示:

\begin{figure}[ht]
  \centering
  \includegraphics[width=\linewidth]{images/string.png}
  \caption{识别字符 Character 和字符串 String Literal 的自动机}
  \label{fig:figure4}
\end{figure}

可以看到,我们在识别字符和字符串的时候,需要特别识别转义字符,因此我们首先构造一个转义字符列表:

\begin{minted}[linenos,frame=lines,framesep=2mm]{python}
# 转义字符
cEscSequence = ['\'', '"', '?', '\\', 'a', 'b', 'f', 'n', 'r', 't', 'v']
\end{minted}

之后,我们就可以根据上面自动机构建识别字符和字符串的代码。

\begin{minted}[linenos,frame=lines,framesep=2mm]{python}
# String!
elif character == '"':
  global stringState
  while index < len(code):
    codeValue = codeValue + character
    if stringState == 0:
      if character == '"':
        stringState = 1
    elif stringState == 1:
      if character == '\\':
        stringState = 3
      elif character == '"':
        stringState = 2
        break
    elif stringState == 2:
      break
    elif stringState == 3:
      if character in cEscSequence:
        stringState = 1
    character = code[index]
    index = index + 1
  if stringState == 2:
    codeType = 'string'
    stringState = 0
  else:
    print('Illegal string.')
    stringState = 0

# Char!
elif character == '\'':
  global charState
  while index < len(code):
    codeValue = codeValue + character
    if charState == 0:
      ...
    character = code[index]
    index = index + 1
  if charState == 2:
    codeType = 'character'
    charState = 0
  else:
    codeType = 'illegal char'
    charState = 0
\end{minted}

\subsubsection{算符(包括运算符 Operator 和界限符 Delimiter)的判断}
在 C 语言中,能够被识别的算符有:

\begin{center}
  ! " \# \$ \% \& ' ( ) * + , - . / : ; < = > ? @ [ \textbackslash\ ] \textasciicircum\ \textunderscore \ ` \{ | \} \textasciitilde
\end{center}

这些算符通过组合可以成为单目或多目运算符,一些单独的算符也属于界限符。通过查阅资料,我找到了 C 语言定义了以下界限符:

\begin{center}
  \captionsetup{position=above}
  \captionof{table}{C 语言定义的界限符 Delimiter}
  \begin{tabular}{|c c c c c c c c c c c|}
    \hline
    [ & ] & ( & ) & \{ & \} & ' & " & , & ; & \textbackslash \\
    \hline
  \end{tabular}
\end{center}

C 语言中合法的运算符有这些:

\begin{center}
  \captionsetup{position=above}
  \captionof{table}{C 语言定义的运算符 Operator}
  \begin{tabular}{|c c c c c c c c c c|}
    \hline
    . & -> & ++ & -- & \& & * & + & - & \textasciitilde & ! \\
    / & \% & << & >> & < & > & <= & >= & == & != \\
    \textasciicircum & | & \&\& & || & ? & : & ... & = & *= & /= \\
    \%=  & += & -= & <<= & >>= & \&= & \textasciicircum = & |= & \# & \#\# \\
   <: & :> & <\% & \%> & \%: & & & & & \\
    \hline
  \end{tabular}
\end{center}

因此,我们可以根据算符能否作为单目运算符来将其如下区分:

\begin{center}
  \captionsetup{position=above}
  \captionof{table}{能构成多目运算符的算符分类}
  \label{tb:table4}
  \begin{tabular}{ c | c c c c c }
    \hline
    \textbf{运算符首字符} & \multicolumn{5}{c}{\textbf{对应的多目运算符}} \\
    \hline
    + & ++ & += & & & \\
    - & -= & -= & & & \\
    < & << & <= & <<= & <: & <\% \\
    > & >> & >= & >>= \\
    = & == \\
    ! & != \\
    \& & \&\& & \& = \\
    | & || & |= \\
    * & *= \\
    / & /= \\
    \% & \%= & \%> & \%: \\
    \textasciicircum & \textasciicircum= \\
    : & :> \\
    \# & \#\# \\
    \hline
  \end{tabular}
\end{center}

那么剩下的算符就是无法直接构成多目运算符的算符了。

具体实现中,我们首先定义运算符列表、可作为多目运算符的算符列表和界限符列表这三个列表:

\begin{minted}[linenos,frame=lines,framesep=2mm]{python}
# 运算符
cOperator = ['+', '-', '&', '*', '~', '!', '/',
             '^', '%', '=', '.', ':', '?', '#', '<', '>', '|', '`']
# 可作为二元运算符首字符的算符
cBinaryOp = ['+', '-', '>', '<', '=', '!',
             '&', '|', '*', '/', '%', '^', '#', ':', '.']
# 界限符
cDelimiter = ['[', ']', '(', ')', '{', '}', '\'', '"', ',', ';', '\\']
\end{minted}

对于界限符,由于在词法分析这一步骤我们尚不需要对括号匹配等内容进行识别,因此我们只需要匹配输入 Token 是否为 界限符即可。具体实现如下:

\begin{minted}[linenos,frame=lines,framesep=2mm]{python}
  # Delimiters
  elif character in cDelimiter:
    codeValue = codeValue + character
    codeType = 'delimiter'
\end{minted}

对于操作符,由于我们需要考虑单目和多目运算符进行匹配工作,因此这里我们根据上面表 \ref{tb:table4} 进行实际的代码实现,构造自动机进行 Token 识别。这里涉及到的自动机非常复杂,也远远超过了我们本次实验报告的内容,这里我就不具体进行介绍了。下面是大致的代码实现:

\begin{minted}[linenos,frame=lines,framesep=2mm]{python}
# Operators
elif character in cOperator:
  global operatorState
  while character in cOperator:
    codeValue = codeValue + character
    if operatorState == 0:
      if not character in cBinaryOp:
        operatorState = 20
        break
      else:
        ...

    character = code[index]
    index = index + 1
  if operatorState >= 2 and operatorState <= 18:
    index = index - 1
    codeType = 'Unary operator'
    operatorState = 0
  elif operatorState == 20:
    codeType = 'Unary operator'
    operatorState = 0
  elif operatorState == 1:
    codeType = 'Multicast operator'
    operatorState = 0
  else:
    index = index - 1
    codeType = 'Illegal operator'
    operatorState = 0
\end{minted}

我们到这里基本实现了所有 Token 类型的识别。

\subsubsection{输出 XML 格式的 Token 识别结果至文件}
最后,我们需要将识别成功的 Token 添加至 XML 树中,并将其输出到指定文件内。为了实现 XML 树的构建,我使用了 \texttt{ElementTree} 的 Python 内置 XML 操作库进行处理,建立根节点 \texttt{project}、Token 节点 \texttt{tokens} 和每个 Token 的属性节点。属性节点大致构建如下:

\begin{minted}[linenos,frame=lines,framesep=2mm]{xml}
<token>
  <numbers>1</numbers>
  <value>int</value>
  <keyword>keyword</keyword>
  <line>1</line>
  <true>true</true>
</token>
\end{minted}

为了将项目名作为根节点 \texttt{project} 的 \texttt{name} 属性,以及确立 XML 输出文件名称,我们首先处理输入文件的名称:

\begin{minted}[linenos,frame=lines,framesep=2mm]{python}
# C source file name
fileName = os.path.basename(filePath)
# XML output file name
xmlFileName = os.path.splitext(fileName)[0] + '.token.xml'
\end{minted}

之后,我们创建 XML 树:

\begin{minted}[linenos,frame=lines,framesep=2mm]{python}
# Create XML tree
xmlTree = ElementTree.Element('project', {
    'name': fileName
})
xmlTokens = ElementTree.SubElement(xmlTree, 'tokens')
\end{minted}

接下来,我们将每次识别的 Token 属性依次赋给每棵子树的节点:

\begin{minted}[linenos,frame=lines,framesep=2mm]{python}
while index <= len(code) - 1:
  ...
  if codeType != '':
    xmlToken = ElementTree.SubElement(xmlTokens, 'token')

    xmlNumber = ElementTree.SubElement(xmlToken, 'numbers')
    xmlValue = ElementTree.SubElement(xmlToken, 'value')
    xmlType = ElementTree.SubElement(xmlToken, 'keyword')
    xmlLine = ElementTree.SubElement(xmlToken, 'line')
    xmlValid = ElementTree.SubElement(xmlToken, 'true')

    xmlNumber.text = str(codeNum)
    xmlValue.text = str(codeValue)
    xmlType.text = codeType.lower()
    xmlLine.text = str(codeLine)
    if not 'illegal' in codeType.lower():
      xmlValid.text = 'true'
    else:
      xmlValid.text = 'false'
    ...
    codeNum = codeNum + 1
\end{minted}

这样,我们就成功的将 Token 的识别结果输出至树 \texttt{xmlTree},构建了 XML 树。最后,我们将 XML 树的内容格式化(Prettify)并赋予 \texttt{utf-8} 的文件编码格式,就可以输出至文件了。

\begin{minted}[linenos,frame=lines,framesep=2mm]{python}
# Turn XML tree to string
xmlString = ElementTree.tostring(xmlTree)
# Set XML indent and encoding (This returns a byte for the file to read)
xml = minidom.parseString(xmlString).toprettyxml(indent='  ', encoding='utf-8')
# Write XML to file
with open(xmlFileName, 'wb') as f:
  f.write(xml)
print('Written XML token processing results to file:', xmlFileName)
\end{minted}

\section{实验结果}

\subsection{Token 的识别结果}
我们在 Token 识别的过程中,将识别到的 Token 属性依次输出,以下面这段 C 语言代码段为例子:

\begin{minted}[linenos,frame=lines,framesep=2mm]{c}
int main(int a, int b)
{
  a = a & b;
  return a + b;
}
\end{minted}

我们可以得到如下图 \ref{fig:subfig1} 的输出结果:

\begin{figure}[h]
  \begin{subfigure}{0.49\textwidth}
    \includegraphics[width=\linewidth]{images/output.png}
    \caption{识别过程输出的 Token 属性}
    \label{fig:subfig1}
  \end{subfigure}
  \begin{subfigure}{0.49\textwidth}
    \includegraphics[width=\linewidth]{images/output2.png}
    \caption{对于不常用 Token 的识别处理}
    \label{fig:subfig2}
  \end{subfigure}
  \caption{词法分析器对 Token 的识别结果}
  \label{fig:figure5}
\end{figure}

为了更加准确的测试我本次实验实现的词法分析器,我准备了一个如下的 C 语言代码段:

\begin{minted}[linenos,frame=lines,framesep=2mm]{c}
int a = 0xA1Eull;
float b = 1.5e-4F;
char *s = "一个很长的字符串";
a += 2;
\end{minted}

我使用我本次实现的词法分析器进行识别,得到了如上图 \ref{fig:subfig2} 的识别结果,可以看到对于复杂的输入串,我们的词法分析器也是准确的。

\subsection{XML 文件的输出结果}

接下来,我们将识别号的 Token 各个属性进行 XML 格式化,并输出至文件中。我成功生成了 XML 文件,并将之和 \texttt{BIT-MiniCC} 词法分析步骤的 XML 文件进行比对,得到了如下图 \ref{fig:figure7} 的 XML 识别结果。

\begin{figure}[h]
  \centering
  \includegraphics[width=\linewidth]{images/XML.png}
  \caption{XML 文件的输出结果和 BIT-MiniCC 输出 XML 对比}
  \label{fig:figure7}
\end{figure}

\section{实验心得体会}
通过本次实验,我重新认识了 C 语言编写的程序在编译过程中词法分析的具体方法。我不仅更加了解了标识符、常量、字符和算符的具体文法描述,还实际的利用了自动机对这些 Token 进行匹配,大致实现了一个简单的词法识别器,能够对一段 C 语言代码进行分析,得到每个识别 Token 的位置、属性和是否合法等性质。

在本次实验的词法分析中,虽然每种 Token 的文法描述都非常复杂,但是我发现,只要认真仔细的写好自动机的具体识别过程,利用 Python 来实现这个自动机还是相对简单。

同时,我也在本次实验中领略到各种不常用的 C 语言合法的语法,比如 \texttt{0777}、\texttt{0xAFull} 等八进制、十六进制整数,比如 \texttt{1.5e-4}、\texttt{0.5F} 等各种形式的浮点数,比如 \texttt{<\%}、\texttt{|=}、\texttt{...} 等不常见的算符。这让我重新了解了 C 语言的语法表示,也让我对自动机理论有了新的认识。
\end{document}