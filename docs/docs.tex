\documentclass[UTF8]{ctexart}
\usepackage[a4paper, top=25.4mm, bottom=25.4mm, left=31.8mm, right=31.8mm]{geometry}
\usepackage{graphicx}
\usepackage{amsmath}
\usepackage{multirow}
\usepackage{subcaption}
\usepackage{tabu}
\usepackage[table]{xcolor}

\setCJKmainfont{Noto Serif CJK SC}
\setlength{\parskip}{1em}
\definecolor{lightergray}{gray}{0.85}

\begin{document}
\begin{titlepage}
  \begin{center}
    \vspace*{1cm}

    \Large
    编译原理

    \vspace{0.5cm}
    \Huge
    \textbf{词法分析实验实验报告}

    \vfill

    \normalsize\kaishu
    班级:07111603 \\
    学号:1120161730 \\
    姓名:武上博 \\
    \today
    \vspace{1cm}
  \end{center}
\end{titlepage}

\tableofcontents
\newpage

\section{实验目的}
\begin{enumerate}
  \item 熟悉 C 语言的词法规则,了解编译器词法分析器的主要功能
  \item 掌握典型词法分析器构造的相关技术和方法,设计并实现 C 语言词法分析器
  \item 掌握编译器从前端到后端各个模块的工作原理,词法分析模块与其他模块之间的交互过程
\end{enumerate}

\section{实验内容}
根据 C 语言的词法规则,设计并识别 C 语言所有单词类的词法分析器的确定有限状态自动机,并使用 Java、C/C++、Python 其中的任意一种语言,采用程序中心法或者数据中心法设计并实现词法分析器。词法分析器的输入为 C 语言源程序,输出为属性字流。

\section{实验的具体过程步骤}

\section{实验结果}

\section{实验心得体会}
\end{document}